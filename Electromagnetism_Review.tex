\documentclass[Chinese, use boldface, twocolumn]{lebhart}

%%================================
%% Import toolkit
%%================================
\usepackage{ProjLib}
\usetikzlibrary{calc}
\usepackage{mathtools}

\usepackage{tabularx}
\usepackage{multicol, multirow}
\usepackage{booktabs}
\usepackage{siunitx}
\usepackage{derivative}

\allowdisplaybreaks

\sisetup{per-mode=symbol}

\newcolumntype{Y}{>{\centering\arraybackslash}X}
\newcolumntype{M}[1]{>{\centering\arraybackslash}m{#1}}
\aboverulesep = 0mm \belowrulesep = 0mm

\UseLanguage{Chinese}

%%================================
%% tip
%%================================
\usepackage[many]{tcolorbox}
\newenvironment{tip}[1][提示]{%
    \begin{tcolorbox}[breakable,
        enhanced,
        width = \linewidth,
        colback = paper, colbacktitle = paper,
        colframe = gray!50, boxrule=0.2mm,
        coltitle = black,
        fonttitle = \sffamily,
        attach boxed title to top left = {yshift=-\tcboxedtitleheight/2, xshift=.5cm},
        boxed title style = {boxrule=0pt, colframe=paper},
        before skip = 0.3cm,
        after skip = 0.3cm,
        top = 3mm,
        bottom = 3mm,
        title={\scshape\sffamily #1}]%
}{\end{tcolorbox}}

\title{电磁场部分复习}
\author{赵舞穹}
\date{\today}

\numberwithin{equation}{section}

\begin{document}

\maketitle

\section{麦克斯韦方程}

\subsection{基本量及其关系}

\begin{itemize}
  \item $\vec{E}(\vec{r},t)$ --- 电场强度 --- \si{\V\per\m}
  \item $\vec{D}(\vec{r},t)$ --- 电通量密度(电位移) --- \si{\C\per\m\squared}
  \item $\vec{H}(\vec{r},t)$ --- 磁场强度 --- \si{\weber\per\m\squared}
  \item $\vec{B}(\vec{r},t)$ --- 磁感应强度(磁通量密度) --- \si{\A\per\m}
\end{itemize}

\begin{tip}
  此处所有的矢量符号我没有按照书本的印刷体,而是和手写一样使用箭头表示($\vec{\cdot}$);
  大家在作业、考试中可以注意书写的规范性,区分矢量和标量.
\end{tip}

\subsubsection{库仑定理}

\begin{equation}
  \vec{E}=\frac{q}{2\pi\varepsilon r^2}\vec{r}_0,
\end{equation}
式中 $\varepsilon$ 为介质常数,定义为
\begin{equation}
  \varepsilon\triangleq\varepsilon_r\varepsilon_0,
\end{equation}
式中 $\varepsilon_r$ 为相对介电系数.

\subsection{洛伦兹力}
运动的点电荷在磁场中受到的力:
\begin{equation}
  \vec{F}=q(\vec{E}+\vec{v}\times\vec{B}).
\end{equation}

\subsubsection{介质特性}
物质的本构关系:
\begin{subequations}
  \begin{align}
    \vec{D}&=\varepsilon\vec{E},\\
    \vec{B}&=\mu\vec{H},\\
    \vec{J}_c&=\sigma\vec{E},
  \end{align}
\end{subequations}
式中传导电流密度 $\vec{J}_c=\vec{J}-\vec{J}_V=\vec{J}-\rho_V\vec{v}$.

\subsection{Maxwell 方程组}

\subsubsection{法拉第定理}
微分形式、积分形式和时谐场的复矢量形式:
\begin{subequations}
  \begin{align}
    \nabla\times\vec{E}&=-\pdv{\vec{B}}{t},\label{eq:Faraday_d}\\
    \oint_l\vec{E}\cdot\odif{\vec{l}}&=-\int_S\pdv{\vec{B}}{t}\cdot\odif{\vec{S}},\\
    \nabla\times\vec{E}&=-j\omega\vec{B}.
  \end{align}
\end{subequations}

\subsubsection{安培定理}
微分形式、积分形式和时谐场的复矢量形式:
\begin{subequations}
  \begin{align}
    \nabla\times\vec{H}&=\vec{J}+\pdv{\vec{D}}{t},\\
    \oint_l\vec{H}\cdot\odif{\vec{l}}&=\int_S\left(\vec{J}+\pdv{\vec{D}}{t}\right)\cdot\odif{\vec{S}},\\
    \nabla\times\vec{H}&=\vec{J}+j\omega\vec{D}.
  \end{align}
\end{subequations}

\subsubsection{高斯定理}
微分形式和积分形式:
\begin{subequations}
  \begin{align}
    \nabla\cdot\vec{D}&=\rho_V,\\
    \oint_S\vec{D}\cdot\odif{\vec{S}}&=\int_V\rho_V\odif{V}.
  \end{align}
\end{subequations}

\subsubsection{磁通连续性原理}
微分形式和积分形式:
\begin{subequations}
  \begin{align}
    \nabla\cdot\vec{B}&=0,\\
    \oint_S\vec{B}\cdot\odif{\vec{S}}&=0.
  \end{align}
\end{subequations}

\subsubsection{坡印廷定理}

坡印廷矢量定义
\begin{equation}
  \vec{S}(\vec{r},t)=\vec{E}(\vec{r},t)\times\vec{H}(\vec{r},t).
\end{equation}
特殊地,对于\textbf{时谐场},有
\begin{align}
  \vec{S}(\vec{r})&=\vec{E}(\vec{r})\times\vec{H}^*(\vec{r}),\\
  \langle\vec{S}(t)\rangle&=\frac12\mathrm{Re}\left[\vec{E}(\vec{r})\times\vec{H}^*(\vec{r})\right].
\end{align}

\section{平面波}

\subsection{波方程}

在无源空间中,我们有:
\begin{equation}
  (\nabla^2+k^2)\left\{\begin{aligned}
    \vec{E}\\\vec{H}
  \end{aligned}\right.=0,
\end{equation}
式中
\begin{equation}
  k^2=\omega^2\mu\varepsilon,
\end{equation}
其中 $|k|=\omega\sqrt{\mu\varepsilon}$ 为传播常数.\ %
在方向上,$\vec{E}$, $\vec{H}$ 和波矢 $\vec{k}$ 两两垂直.

波阻抗(本征阻抗)
\begin{equation}
  Z=\frac1Y=\frac{\omega\mu}{k}=\frac{k}{\omega\varepsilon}=\sqrt{\frac{\mu}{\varepsilon}},
\end{equation}
波长
\begin{equation}
  \lambda=\frac{2\pi}{k},
\end{equation}
相速
\begin{equation}
  v_p=\frac{\omega}{k}=\frac{\omega}{\omega\sqrt{\mu\varepsilon}}=\frac1{\sqrt{\mu\varepsilon}}.
\end{equation}

\subsection{极化}
对于沿 $\vec{z}$ 方向的电场
\begin{equation}
  \vec{E}\triangleq\vec{E}_x+\vec{E}_y,
\end{equation}
判断
\begin{equation}
  \dot{A}=\frac{\dot{E}_x}{\dot{E}_y}=Ae^{j\varphi},
\end{equation}
的幅度 $A$ 和相位 $\varphi$.

\subsection{有耗介质}

定义复介电常数
\begin{equation}
  \widetilde{\varepsilon}\triangleq\varepsilon-j\frac{\sigma}{\omega},
\end{equation}
其中虚部表示介质电导率的影响.\ %
穿透深度
\begin{equation}
  d_p=\frac1{k_i},
\end{equation}
此处场振幅减为 $z=0$ 处的 $\frac1e$.

\subsubsection{电导率很小}
若满足 $\frac{\sigma}{\omega\varepsilon}\ll1$,有
\begin{gather}
  k=\omega\sqrt{\mu\varepsilon}\left(1-j\frac{\sigma}{2\omega\varepsilon}\right),\\
  d_p=\frac2\sigma\sqrt{\frac\varepsilon\mu}.
\end{gather}

\subsubsection{电导率很大}
若满足 $\frac{\sigma}{\omega\varepsilon}\gg1$,有
\begin{gather}
  k=\sqrt{\omega\mu\frac\sigma2}(1-j),\\
  d_p=\frac{2}{\omega\mu\sigma}=\delta,
\end{gather}
其中 $\delta$ 很小,被称为趋肤深度.

\section{波的反射与折射及多层介质中波的传播}

To be added ...

\appendix
\section{作业讨论}

\subsection{习题3.4}
\begin{problem*}[]
  如果在某一表面 $\vec{E}=\vec{0}$,是否就可得出在该表面 $\pdv{\vec{B}}{t}=0$?为什么?
\end{problem*}

\textbf{不能}.\ %
由 \eqref{eq:Faraday_d},
\begin{equation}
  \pdv{\vec{B}}{t}=-\nabla\times\vec{E}\neq0.
\end{equation}
\begin{tip}[注意]
  零向量的旋度(curl)不一定为零!
  我们可以具体将其写为
  \begin{equation}
    \nabla\times\vec{E}=\begin{vmatrix}
      \vec{x}_0&\vec{y}_0&\vec{z}_0\\
      \pdv{}{x}&\pdv{}{y}&\pdv{}{z}\\
      E_x&E_y&E_z
    \end{vmatrix}\neq0.
  \end{equation}
\end{tip}


\section{Open Access License}
此文档依据 \href{https://github.com/Teddy-van-Jerry/Electromagnetism_Review_cn/blob/master/LICENSE}{CC--BY--SA--4.0} 公开.

\end{document}
